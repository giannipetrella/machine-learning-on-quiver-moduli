\usepackage{fontenc}
\usepackage{graphicx, float}
\usepackage{geometry}
\usepackage{parskip}

\usepackage{colonequals}
\usepackage{booktabs}
\usepackage{amsmath}
\usepackage{amsthm, amsfonts, amssymb}
\usepackage{minted}

% \usepackage[implicit=true]{hyperref}
% \hypersetup{
% hypertexnames = false,
% bookmarksdepth = 2,
% bookmarksopen = true,
% colorlinks,
% linkcolor = blue,
% citecolor = green,
% urlcolor = blue,
% pdfstartview={XYZ null null 1}
% }
\usepackage{bookmark}
% \usepackage[capitalise]{cleveref}

\usepackage{tikz, tikz-cd}
\usetikzlibrary{calc}
\usetikzlibrary{arrows}
\usetikzlibrary{matrix}
\usetikzlibrary{positioning}
\usetikzlibrary{arrows.meta}
% fonts

% fonts
\usepackage[T1]{fontenc}
\usepackage{libertine}
\usepackage[libertine]{newtxmath}
%\usepackage[scaled=0.83]{beramono}
\usepackage[scaled=0.8]{FiraMono}
\renewcommand*\ttdefault{lmvtt} %monospace font
\usepackage{microtype}
\frenchspacing

%\usepackage[backend=biber, datamodel=mrnumber, maxbibnames=99, sortcites]{biblatex}
%\addbibresource{bibliography.bib}

\usepackage{gitinfo2}

\newcommand\gitfootnote[1]{
  \begin{NoHyper}
  \renewcommand\thefootnote{}\footnote{#1}
  \addtocounter{footnote}{-1}
  \end{NoHyper}
}


% \usepackage{thmtools}
% \declaretheoremstyle[
%   spaceabove = 3pt,
%   spacebelow = 3pt,
%   bodyfont = \itshape,
% ]{first}
% \declaretheoremstyle[
%   spaceabove = 3pt,
%   spacebelow = 3pt,
% ]{second}
% \declaretheorem[numberwithin=section, style=first]{theorem}
% \declaretheorem[sibling=theorem, style=first]{conjecture}
% \declaretheorem[sibling=theorem, style=first]{corollary}
% \declaretheorem[sibling=theorem, style=first]{lemma}
% \declaretheorem[sibling=theorem, style=first]{proposition}

% \declaretheorem[sibling=theorem, style=second]{example}
% \declaretheorem[sibling=theorem, style=second]{remark}
% \declaretheorem[sibling=theorem, style=second]{definition}
% \declaretheorem[sibling=theorem, style=second]{notation}
% \declaretheorem[sibling=theorem, style=second]{assumption}
% \declaretheorem[sibling=theorem, style=second]{convention}
% \declaretheorem[sibling=theorem, style=second]{setup}

% \Crefname{assumption}{Assumption}{Assumptions}
% \Crefname{convention}{Convention}{Conventions}
% \Crefname{setup}{Setup}{Setups}

% \declaretheorem[numberwithin=section, style=first, title=Theorem]{alphatheorem}
% \declaretheorem[sibling=alphatheorem, style=first, title=Conjecture]{alphaconjecture}
% \declaretheorem[sibling=alphatheorem, style=first, title=Question]{alphaquestion}

% \renewcommand{\thealphatheorem}{\Alph{alphatheorem}}
% \renewcommand{\thealphaconjecture}{\Alph{alphaconjecture}}
% \renewcommand{\thealphaquestion}{\Alph{alphaquestion}}
% \crefname{alphatheorem}{Theorem}{Theorems}
% \crefname{alphaconjecture}{Conjecture}{Conjectures}
% \crefname{alphaquestion}{Question}{Questions}



\DeclareMathOperator{\Hom}{Hom}
\DeclareMathOperator{\Ext}{Ext}
\DeclareMathOperator{\GL}{GL}
\DeclareMathOperator{\repspace}{R}
\DeclareMathOperator{\modulispace}{M}
\DeclareMathOperator{\Pic}{Pic}
\DeclareMathOperator{\rk}{rk}
\DeclareMathOperator{\Hochschild}{HH}
\DeclareMathOperator{\HH}{H}
\DeclareMathOperator{\hh}{h}


% Numbers
\newcommand{\NN}{\mathbb{N}}
\newcommand{\ZZ}{\mathbb{Z}}
\newcommand{\QQ}{\mathbb{Q}}
\newcommand{\RR}{\mathbb{R}}
\newcommand{\CC}{\mathbb{C}}

% Spaces
\newcommand{\PP}{\mathbb{P}}
% \newcommand{\AA}{\mathbb{A}}


% Sheaves
\DeclareMathOperator{\sheafHom}{\mathscr{H}\kern -2.5pt\mathit{om}}
\DeclareMathOperator{\sheafExt}{\mathscr{E}\mathit{xt}}


% If and only if
\newcommand{\iif}{\ensuremath{\Leftrightarrow}}

% Arrow with optional label.
% Use as A \to[label] B
\renewcommand*{\to}[1][]{\overset{#1}{\rightarrow}}


\newcommand{\stable}{\hyphen\mathrm{st}}
\newcommand{\semistable}{\hyphen\mathrm{sst}}
\newcommand{\semisimple}{\hyphen\mathrm{ssimp}}
\newcommand{\tuple}[1]{\ensuremath{\mathbf{#1}}}
\newcommand{\quivertools}{{\tt{QuiverTools}}}
\newcommand{\gitquot}{/\kern-0.2em /}
\newcommand{\GLd}{\ensuremath{\operatorname{GL}_{\tuple{d}}}}
