\documentclass[final]{beamer}
\usepackage{fontenc}
\usepackage{graphicx, float}
\usepackage{geometry}
\usepackage{parskip}

\usepackage{colonequals}
\usepackage{booktabs}
\usepackage{amsmath}
\usepackage{amsthm, amsfonts, amssymb}

% \usepackage[implicit=true]{hyperref}
% \hypersetup{
% hypertexnames = false,
% bookmarksdepth = 2,
% bookmarksopen = true,
% colorlinks,
% linkcolor = blue,
% citecolor = green,
% urlcolor = blue,
% pdfstartview={XYZ null null 1}
% }
\usepackage{bookmark}
% \usepackage[capitalise]{cleveref}

\usepackage{tikz, tikz-cd}
\usetikzlibrary{calc}
\usetikzlibrary{arrows}
\usetikzlibrary{matrix}
\usetikzlibrary{positioning}

% fonts

% fonts
\usepackage[T1]{fontenc}
\usepackage{libertine}
\usepackage[libertine]{newtxmath}
%\usepackage[scaled=0.83]{beramono}
\usepackage[scaled=0.8]{FiraMono}
\renewcommand*\ttdefault{lmvtt} %monospace font
\usepackage{microtype}
\frenchspacing

%\usepackage[backend=biber, datamodel=mrnumber, maxbibnames=99, sortcites]{biblatex}
%\addbibresource{bibliography.bib}

\usepackage{gitinfo2}

\newcommand\gitfootnote[1]{
  \begin{NoHyper}
  \renewcommand\thefootnote{}\footnote{#1}
  \addtocounter{footnote}{-1}
  \end{NoHyper}
}


% \usepackage{thmtools}
% \declaretheoremstyle[
%   spaceabove = 3pt,
%   spacebelow = 3pt,
%   bodyfont = \itshape,
% ]{first}
% \declaretheoremstyle[
%   spaceabove = 3pt,
%   spacebelow = 3pt,
% ]{second}
% \declaretheorem[numberwithin=section, style=first]{theorem}
% \declaretheorem[sibling=theorem, style=first]{conjecture}
% \declaretheorem[sibling=theorem, style=first]{corollary}
% \declaretheorem[sibling=theorem, style=first]{lemma}
% \declaretheorem[sibling=theorem, style=first]{proposition}

% \declaretheorem[sibling=theorem, style=second]{example}
% \declaretheorem[sibling=theorem, style=second]{remark}
% \declaretheorem[sibling=theorem, style=second]{definition}
% \declaretheorem[sibling=theorem, style=second]{notation}
% \declaretheorem[sibling=theorem, style=second]{assumption}
% \declaretheorem[sibling=theorem, style=second]{convention}
% \declaretheorem[sibling=theorem, style=second]{setup}

% \Crefname{assumption}{Assumption}{Assumptions}
% \Crefname{convention}{Convention}{Conventions}
% \Crefname{setup}{Setup}{Setups}

% \declaretheorem[numberwithin=section, style=first, title=Theorem]{alphatheorem}
% \declaretheorem[sibling=alphatheorem, style=first, title=Conjecture]{alphaconjecture}
% \declaretheorem[sibling=alphatheorem, style=first, title=Question]{alphaquestion}

% \renewcommand{\thealphatheorem}{\Alph{alphatheorem}}
% \renewcommand{\thealphaconjecture}{\Alph{alphaconjecture}}
% \renewcommand{\thealphaquestion}{\Alph{alphaquestion}}
% \crefname{alphatheorem}{Theorem}{Theorems}
% \crefname{alphaconjecture}{Conjecture}{Conjectures}
% \crefname{alphaquestion}{Question}{Questions}



\DeclareMathOperator{\Hom}{Hom}
\DeclareMathOperator{\Ext}{Ext}
\DeclareMathOperator{\GL}{GL}
\DeclareMathOperator{\repspace}{R}
\DeclareMathOperator{\modulispace}{M}
\DeclareMathOperator{\Pic}{Pic}
\DeclareMathOperator{\rk}{rk}
\DeclareMathOperator{\Hochschild}{HH}
\DeclareMathOperator{\HH}{H}
\DeclareMathOperator{\hh}{h}


% Numbers
\newcommand{\NN}{\mathbb{N}}
\newcommand{\ZZ}{\mathbb{Z}}
\newcommand{\QQ}{\mathbb{Q}}
\newcommand{\RR}{\mathbb{R}}
\newcommand{\CC}{\mathbb{C}}

% Spaces
\newcommand{\PP}{\mathbb{P}}
% \newcommand{\AA}{\mathbb{A}}


% Sheaves
\DeclareMathOperator{\sheafHom}{\mathscr{H}\kern -2.5pt\mathit{om}}
\DeclareMathOperator{\sheafExt}{\mathscr{E}\mathit{xt}}


% If and only if
\newcommand{\iif}{\ensuremath{\Leftrightarrow}}

% Arrow with optional label.
% Use as A \to[label] B
\renewcommand*{\to}[1][]{\overset{#1}{\rightarrow}}


\newcommand{\stable}{\hyphen\mathrm{st}}
\newcommand{\semistable}{\hyphen\mathrm{sst}}
\newcommand{\semisimple}{\hyphen\mathrm{ssimp}}
\newcommand{\tuple}[1]{\ensuremath{\mathbf{#1}}}
\newcommand{\quivertools}{{\tt{QuiverTools}}}
\newcommand{\gitquot}{/\kern-0.2em /}
\newcommand{\GLd}{\ensuremath{\operatorname{GL}_{\tuple{d}}}}


\newcommand{\todo}[1]{{\color{blue}#1}}
% ====================
% Packages
% ====================

% \usepackage[T1]{fontenc}
% \usepackage{lmodern}
\usepackage[size=custom, width=76.2, height=101.6, scale=1.0]{beamerposter}
\usetheme{gemini}
\usecolortheme{gemini}
% \usepackage{graphicx}
\usepackage{epstopdf}  % Include the epstopdf package
\usepackage{pgfplots}
\pgfplotsset{compat=1.14}
\usepackage{anyfontsize}
\usepackage{booktabs}


\usepackage[backend=biber, datamodel=mrnumber, maxbibnames=99, sortcites]{biblatex}
\addbibresource{poster.bib}

% ====================
% Lengths
% ====================

% If you have N columns, choose \sepwidth and \colwidth such that
% (N+1)*\sepwidth + N*\colwidth = \paperwidth
\newlength{\sepwidth}
\newlength{\colwidth}
\setlength{\sepwidth}{0.025\paperwidth}
\setlength{\colwidth}{0.3\paperwidth}

\newcommand{\separatorcolumn}{\begin{column}{\sepwidth}\end{column}}

% ====================
% Title
% ====================

\title{Machine learning features of quiver moduli}

\author{Gianni Petrella \inst{1}}

\institute[shortinst]{\inst{1} University of Luxembourg}
% ====================
% Footer (optional)
% ====================

\footercontent{
  \href{https://giannipetrella.eu}{www.giannipetrella.eu} \hfill
  Princeton ML Theory Summer School - August 12-21, 2025}
  % Computations in Algebra and Geometry, ETH - August 25-29, 2025}

% ====================
% Logo (optional)
% ====================

% Include Eurecom logo on the right side of the header
\logoright{\includegraphics[height=7cm]{UNI-Logo-en-rgb.png}}
% Include Lab logo on the left side of the header
% \logoleft{\includegraphics[height=7cm]{./logos/s3logo.png}}

% ====================
% Body
% ====================

\begin{document}

\begin{frame}[t]
\begin{columns}[t]
\separatorcolumn

\begin{column}{\colwidth}

  \begin{block}{Representations of quivers}

  A \emph{quiver} $Q$ is a finite directed multigraph with vertices $Q_0$ and arrows $Q_1$.
  A \emph{representation} $V$ of $Q$ is a choice of vector space $V_i$ per vertex $i$
  and a choice of linear application $M_{\alpha}$ for each arrow $\alpha$ \cite{2311.17003}.

  The \emph{dimension vector} of a representation $V$
  is~$\tuple{d} \colonequals \dim(V) \colonequals (\dim(V_i))_{i \in Q_0}$.

    \begin{figure}
      \centering
      \begin{tikzpicture}
        \node (1) at (0,0)              {$\bullet$};
        \node (2) at (8,0)              {$\bullet$};

        \draw[->, bend left  = 30] (1) edge (2);
        \draw[->, bend left  = 10] (1) edge (2);
        \draw[->, bend right = 10] (1) edge (2);
        \draw[->, bend right = 30] (1) edge (2);
      \end{tikzpicture}
      \caption{The $4$-Kronecker quiver.}
    \end{figure}

    Two representations are \emph{isomorphic} if they are equivalent up to
    a change of basis -
    that is, if they lie in the same orbit of the action of $\operatorname{GL_{\tuple{d}}}$.

    When choosing a \emph{stability parameter} $\theta \in \ZZ^{Q_0}$ for which
    $\theta \cdot \tuple{d} = 0$, we say that a representation $V$
    is $\theta$-stable, respectively $\theta$-semistable,
    if all of its subrepresentations $W$ satisfy
    $\theta \cdot \dim(W) < 0$, respectively
    $\theta\cdot\dim(W)\leq 0$.

    \todo{why are quiver moduli useful}
    \end{block}

  \begin{block}{Moduli spaces of stable representations}

    The \emph{moduli space of representations} of $Q$ of dimension vector $\tuple{d}$
    is a variety whose points correspond to isomorphism classes
    of $\theta$-stable representations.

    \todo{proprties of quiver moduli, citations}

  \end{block}

  \begin{alertblock}{Geometric invariants}

    Many geometric invariants of quiver moduli can be computed effectively
    using the software package \quivertools \cite{quivertools}.
    However, complexity is often a limiting factor,
    so for large scale applications we attempt to model
    several of these features using machine learning techniques.

    \begin{itemize}
      \item \textbf{Euler characteristic} is computed via Betti numbers computations \cite{}.

      \item \textbf{(Strong) ample stability} are properties verified by enumerating
      all possible Harder--Narasimhan types and destabilizing subdimension vectors as in \cite{2311.17003}.

      \item \textbf{Teleman inequality bounds} are computed again
      for each Harder--Narasimhan type using \cite{}
    \end{itemize}

  \end{alertblock}

\end{column}

\separatorcolumn

\begin{column}{\colwidth}

  \begin{block}{Applications}

    PatternBoost is a ML-enabled mathematical research workflow presented in \cite{}.
    It involves a training step,
    in which a neural network is trained to produce examples of mathematical objects
    that aim to optimize a mathematically meaningful loss function, and a search phase,
    in which examples produced by the trained model are ``improved'' by standard algorithms
    to further optimise the loss function.
    The process then repeats, and the resulting examples are used to train a model again.

    \begin{enumerate}
      \item \textbf{Morbi mauris purus}, egestas at vehicula et, convallis
        accumsan orci. Orci varius natoque penatibus et magnis dis parturient
        montes, nascetur ridiculus mus.
      \item \textbf{Cras vehicula blandit urna ut maximus}. Aliquam blandit nec
        massa ac sollicitudin. Curabitur cursus, metus nec imperdiet bibendum,
        velit lectus faucibus dolor, quis gravida metus mauris gravida turpis.
      \item \textbf{Vestibulum et massa diam}. Phasellus fermentum augue non
        nulla accumsan, non rhoncus lectus condimentum.
    \end{enumerate}

  \end{block}

  \begin{block}{Fusce Aliquam Magna Velit}

    Et rutrum ex euismod vel. Pellentesque ultricies, velit in fermentum
    vestibulum, lectus nisi pretium nibh, sit amet aliquam lectus augue vel
    velit. Suspendisse rhoncus massa porttitor augue feugiat molestie. Sed
    molestie ut orci nec malesuada. Sed ultricies feugiat est fringilla
    posuere.

    \begin{figure}
      \centering
      \begin{tikzpicture}
        \begin{axis}[
            scale only axis,
            no markers,
            domain=0:2*pi,
            samples=100,
            axis lines=center,
            axis line style={-},
            ticks=none]
          \addplot[red] {sin(deg(x))};
          \addplot[blue] {cos(deg(x))};
        \end{axis}
      \end{tikzpicture}
      \caption{Another figure caption.}
    \end{figure}

  \end{block}

  \begin{block}{Nam Cursus Consequat Egestas}

    Nulla eget sem quam. Ut aliquam volutpat nisi vestibulum convallis. Nunc a
    lectus et eros facilisis hendrerit eu non urna. Interdum et malesuada fames
    ac ante \textit{ipsum primis} in faucibus. Etiam sit amet velit eget sem
    euismod tristique. Praesent enim erat, porta vel mattis sed, pharetra sed
    ipsum. Morbi commodo condimentum massa, \textit{tempus venenatis} massa
    hendrerit quis. Maecenas sed porta est. Praesent mollis interdum lectus,
    sit amet sollicitudin risus tincidunt non.

    Etiam sit amet tempus lorem, aliquet condimentum velit. Donec et nibh
    consequat, sagittis ex eget, dictum orci. Etiam quis semper ante. Ut eu
    mauris purus. Proin nec consectetur ligula. Mauris pretium molestie
    ullamcorper. Integer nisi neque, aliquet et odio non, sagittis porta justo.

    \begin{itemize}
      \item \textbf{Sed consequat} id ante vel efficitur. Praesent congue massa
        sed est scelerisque, elementum mollis augue iaculis.
        \begin{itemize}
          \item In sed est finibus, vulputate nunc gravida, pulvinar lorem. In maximus nunc dolor, sed auctor eros
            porttitor quis.
          \item Fusce ornare dignissim nisi. Nam sit amet risus vel lacus
            tempor tincidunt eu a arcu.
          \item Donec rhoncus vestibulum erat, quis aliquam leo
            gravida egestas.
        \end{itemize}
      \item \textbf{Sed luctus, elit sit amet} dictum maximus, diam dolor
        faucibus purus, sed lobortis justo erat id turpis.
      \item \textbf{Pellentesque facilisis dolor in leo} bibendum congue.
        Maecenas congue finibus justo, vitae eleifend urna facilisis at.
    \end{itemize}

  \end{block}

\end{column}

\separatorcolumn

\begin{column}{\colwidth}

  \begin{exampleblock}{A Highlighted Block Containing Some Math}

    A different kind of highlighted block.

    $$
    \int_{-\infty}^{\infty} e^{-x^2}\,dx = \sqrt{\pi}
    $$

    Interdum et malesuada fames $\{1, 4, 9, \ldots\}$ ac ante ipsum primis in
    faucibus. Cras eleifend dolor eu nulla suscipit suscipit. Sed lobortis non
    felis id vulputate.

    \heading{A Heading Inside a Block}

    Praesent consectetur mi $x^2 + y^2$ metus, nec vestibulum justo viverra
    nec. Proin eget nulla pretium, egestas magna aliquam, mollis neque. Vivamus
    dictum $\mathbf{u}^\intercal\mathbf{v}$ sagittis odio, vel porta erat
    congue sed. Maecenas ut dolor quis arcu auctor porttitor.

    \heading{Another Heading Inside a Block}

    Sed augue erat, scelerisque a purus ultricies, placerat porttitor neque.
    Donec $P(y \mid x)$ fermentum consectetur $\nabla_x P(y \mid x)$ sapien
    sagittis egestas. Duis eget leo euismod nunc viverra imperdiet nec id
    justo.

  \end{exampleblock}

  \begin{block}{Nullam Vel Erat at Velit Convallis Laoreet}

    Class aptent taciti sociosqu ad litora torquent per conubia nostra, per
    inceptos himenaeos. Phasellus libero enim, gravida sed erat sit amet,
    scelerisque congue diam. Fusce dapibus dui ut augue pulvinar iaculis.

    \begin{table}
      \centering
      \begin{tabular}{l r r c}
        \toprule
        \textbf{First Column} & \textbf{Second Column} & \textbf{Third Column} & \textbf{Fourth} \\
        \midrule
        Foo & 13.37 & 384,394 & $\alpha$ \\
        Bar & 2.17 & 1,392 & $\beta$ \\
        Baz & 3.14 & 83,742 & $\delta$ \\
        Qux & 7.59 & 974 & $\gamma$ \\
        \bottomrule
      \end{tabular}
      \caption{A table caption.}
    \end{table}

    Donec quis posuere ligula. Nunc feugiat elit a mi malesuada consequat. Sed
    imperdiet augue ac nibh aliquet tristique. Aenean eu tortor vulputate,
    eleifend lorem in, dictum urna. Proin auctor ante in augue tincidunt
    tempor. Proin pellentesque vulputate odio, ac gravida nulla posuere
    efficitur. Aenean at velit vel dolor blandit molestie. Mauris laoreet
    commodo quam, non luctus nibh ullamcorper in. Class aptent taciti sociosqu
    ad litora torquent per conubia nostra, per inceptos himenaeos.

    Nulla varius finibus volutpat. Mauris molestie lorem tincidunt, iaculis
    libero at, gravida ante. Phasellus at felis eu neque suscipit suscipit.
    Integer ullamcorper, dui nec pretium ornare, urna dolor consequat libero,
    in feugiat elit lorem euismod lacus. Pellentesque sit amet dolor mollis,
    auctor urna non, tempus sem.

  \end{block}

  \begin{block}{References}

  \printbibliography

  \end{block}

\end{column}

\separatorcolumn
\end{columns}
\end{frame}

\end{document}
